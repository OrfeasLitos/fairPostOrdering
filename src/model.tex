\section{Model}
  \subsection{Properties}
    \subsubsection*{Players}
      The are $N \in \mathbb{N}^{*}$ players in the model. A player $u$ is
      defined by her Steem Power $SP \in \mathbb{N}$, her Voting Power $VP \in
      \left[0, 1\right]$, her Likeability Distribution $L \in \mathcal{D}
      \left(\left[0, 1\right]^N\right)$ and her Strategy $S \in \left\{H,
      G\right\} \times \times 2^{\left[N\right]}$.\footnote{We denote the
      powerset of a set $A$ with $2^A$.} Let the $i$th player be represented by
      the tuple $u_i = \left(SP_i, VP_i, L_i, S_i\right)$. The tuple of players
      is defined as $\mathcal{U} = \left(u_1, \dots, u_i, \dots, u_N\right)$ and
      each player is identified by her index in $\mathcal{U}$.

      We will now explain each field in $u_i$ in detail:
      \begin{itemize}
        \item \textbf{Steem Power.} The Steem Power of $u_i$ is defined as
        $SP_{i} \in \mathbb{N}$ and represent the influence of the player in the
        platform. The vector of Steem Power funds for the $N$ players is defined
        as $\mathcal{SP} = \left(SP_{1}, \dots, SP_{i}, \dots, SP_{n}\right)$.
        \item \textbf{Voting Power.} The Voting Power of $u_i$ is defined as
        $VP_i \in \left[0, 1\right]$ and can be understood as  voting influence
        that is used up when voting and regenerates with time. The vector of
        Voting Power for the N players is defined as $\mathcal{VP} = \left(VP_1,
        \dots, VP_i, \dots, VP_N\right)$.

        \item \textbf{Likeability Distribution.} The Likeability Distribution
        $L_i\in \mathcal{D}\left([0,1]^N\right)$\footnote{We denote the set of
        all probability distributions on set $A$ as $\mathcal{D\left(A\right)}$.}
        of $u_i$ is a distribution on how likeable is the content produced by
        $u_i$ to the rest of the players. The Likeability Distribution for the
        whole system is $\mathcal{L} = \left(L_1, \dots, L_i, \dots, L_n\right)$.

        \item \textbf{Strategy.} The strategy of $u_i$ is defined as $S_i \in
        \lbrace H, G \rbrace \times \mathbb{N}^* \times 2^{\left[N\right]}$,
        where the first element is the player's core strategy, the second is her
        attention span and the third is her voting ring.
        \begin{itemize}
          \item \textbf{Honest/Greedy.} $H$ corresponds to the $honest$ and $G$
          to the $greedy$ strategy. An $honest$ player votes according to the
          likeability of a post $l_i$ (defined later)\footnote{TODO: untangle},
          that is to say she votes the posts she likes. For $honest$ players, the
          value of the vote is computed as $v_{H,i} = VP \cdot l \cdot SP$, where
          $l$ is drawn from the Likeability distribution\footnote{TODO: move to
          posts section}. In Steem terms, $l$ can be understood as the weight of
          a vote.

          A $greedy$ player only votes for posts produced by users of its
          Voting Ring. The value of vote for a player if $u_i$ is $greedy$ is
          defined as $v_{G,i} = VP_i \cdot SP_i$,\footnote{TODO: same} as in our
          model all $greedy$ votes are executed with full weight.

          \item \textbf{Attention Span.} This is a positive integer that
          represents the number of posts a player can consider voting
          simultaneously. For the benefit of simplicity, we will assume that this
          number is constant throughought all players.\footnote{TODO: discuss}

          \item \textbf{Voting Ring.}  If player $u_i$ is $honest$, her Voting
          Ring is $R_i = \emptyset$. If $u_i$ is $greedy$, her Voting Ring is $R_i
          \in 2^{\left[N\right]}$. A voting ring is defined as $R_i = \lbrace
          g_1, \dots, g_j, \dots, g_n \rbrace$ where $g_j \in \mathcal{U}$ is the
          $j$th member of the voting ring and $n$ is the size of the ring. Two
          $greedy$ players will either have the same or disjoint voting rings
          ($\forall i \neq j \in \left[N\right], R_i = R_j \vee \left(R_i \cap
          R_j = \emptyset\right)$).
        \end{itemize}
        The tuple of the strategies for the $N$ players is defined as
        $\mathcal{S} = (S_1, \dots, S_i, \dots, s_N)$.
      \end{itemize}
      The set of players is defined as $\mathcal{U} = ( u_1,..., u_i,..., u_n )$
      $\forall i \in \left[N\right]$.\footnote{TODO: fix appearance}

    \subsubsection*{Posts}
      A post is defined as $p = \left(i, l, v\right)$, with $i \in
      \left[N\right], l \sim L_i$ and $v \in \mathbb{R}_{+}$.
      \begin{itemize}
        \item \textbf{Author.} Each player $u_i$ creates one post. The first
        element of that post is the index of its creator, $i$.

        \item \textbf{Likeability.} The likeability of a post is defined as $l
        \in \left[0, 1\right]^N$, where $l$ is drawn from $L_i$ (the Likeability
        Distribution of its creator $u_i$).

        \item \textbf{Votes.} A post has an associated ``vote'' value, which is a
        real non-negative number. It is initialized at 0 and increases whenever a
        player votes for the post, as explained later in detail.
      \end{itemize}
      Let $l_i \sim L_i$ and $p_i = \left(i, l_i, 0\right)$. The set of all posts
      is $\mathcal{P} = \Vert_{i=1}^N p_i$.\footnote{$a \Vert b$ denotes the
      concatenation of $a$ and $b$.}


  \subsection{Game Execution}
    \begin{algorithm}
      \caption{Each player creates one post}
      \label{alg:postGen}
      \begin{algorithmic}[1]
      \Function{GeneratePosts}{$\mathcal{U}$}
        \State $\mathcal{P} = \emptyset$ \Comment{List of posts}
        \For{$u_i \in \mathcal{U}$}
           \State $l \xleftarrow{r} L_i$
             \Comment{Get likeability of posts}
           \State $\mathcal{P} \leftarrow \mathcal{P} \Vert \left(i, l, 0\right)$
             \Comment{Add post to list of Posts}
        \EndFor
        \State $\mathcal{P} \leftarrow \textsc{Shuffle}(\mathcal{P})$
          \Comment{Shuffle posts to a random order}
        \State \Return $\mathcal{P}$
      \EndFunction
      \end{algorithmic}
    \end{algorithm}

    \begin{algorithm}[H]
      \caption{Player casts votes according to her strategy and until she
      reaches her Min Voting Power}
      \label{alg:vote}
      \begin{algorithmic}[1]
      \Function{Vote}{$player,\mathcal{P},s,sp$}
        \Switch{$s$}
          \Case{$Honest$}
            \For{$p \in \mathcal{P}$}
              \Comment{Iterate over all the posts}
              \State
                \Comment{If user likes the post and has not reached min VPower}
              \If{$l_p > 0 $ $\land $ $p.VPower > s.Min$}
                \State $voteValue \leftarrow p.VPower \cdot l_p \cdot sp$
                \State $p \leftarrow p.votes +  voteValue$
              \EndIf
            \EndFor
          \EndCase

          \Case{$Greedy$}
             \For{$p \in \mathcal{P}$}
               \Comment{Iterate over all the posts}
               \State
                 \Comment{If post belongs to voting ring and not reached min
                 VPower}
               \If{$p \in s.R $ $\land $ $p.VPower > s.Min$}
                 \State $voteValue \leftarrow p.VPower \cdot weight \cdot sp$
                 \State $p \leftarrow p.votes +  voteValue$
               \EndIf
             \EndFor
          \EndCase
        \EndSwitch

        \State $\mathcal{P} \leftarrow \textsc{Order}(\mathcal{P}.votes)$
          \Comment{Order posts by value of votes received}
        \State \Return $\mathcal{P}$
      \EndFunction
      \end{algorithmic}
    \end{algorithm}

    \begin{algorithm}[H]
      \caption{Players cast votes for $r$ rounds}
      \label{alg:curation}
      \begin{algorithmic}[1]
      \Function{Curate}{$\mathcal{U}, \mathcal{P}, r$}
        \For{$j = 1$ to $r$}
          \Comment{$r$ voting rounds}
          \For {$u_i \in \mathcal{U}$}
            \If{\textsc{IsVoteRound}$\left(r, j, S_i\right)$}
              \State $\mathcal{P} \leftarrow \textsc{Vote}\left(u_i,
              \mathcal{P}\right)$
                \Comment{Player $i$ votes zero or one posts}
            \EndIf
          \EndFor
          \State $\mathcal{P} \leftarrow \textsc{Order}\left(\mathcal{P}\right)$
            \Comment{Order posts by vote count after each round of votes}
        \EndFor
        \State \Return $\mathcal{P}$
      \EndFunction
      \end{algorithmic}
    \end{algorithm}

    \begin{algorithm}[H]
      \caption{Calculates whether voting in this round is optimal}
      \label{alg:isvoteround}
      \begin{algorithmic}[1]
    \begin{algorithm}[H]
      \caption{Posts curation procedure}
      \label{alg:main}
      \hspace*{\algorithmicindent} \textbf{Input:} $\mathcal{U}, r$ \\
      \hspace*{\algorithmicindent} \textbf{Output:} $\mathcal{P}$
      \begin{algorithmic}[1]
      \State $\mathcal{P} \leftarrow
      \textsc{GeneratePosts}\left(\mathcal{U}\right)$
      \State $\mathcal{P} \leftarrow \textsc{Curate}\left(\mathcal{U},
      \mathcal{P}, r\right)$
      \State \Return $\mathcal{P}$
      \end{algorithmic}
    \end{algorithm}
