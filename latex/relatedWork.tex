\section{Related Work}
User-generated content (UGC) has been identified as a fundamental component of social media platforms and Web 2.0 in general~\cite{kaplan2010users}. The content created by users needs to be curated, and crowdsourced content curation~\cite{askalidis2013theoretical} has emerged as an alternative to expert-based~\cite{stanoevska2012content} or algorithmic-based~\cite{rader2015understanding} curation techniques. Motivated by the widespread adoption of crowdsourced aggregation sites such as Reddit or Digg\footnote{\url{http://digg.com/} Accessed: 2018-09-25}, several research efforts~\cite{das2010ranking,ghosh2011incentivizing,abbassi2014distributed} have aimed to model the mechanics and incentives for users in UGC platforms. This surge of interest is accompanied by empirical studies which have shown how
social media users behave strategically when they publish and consume content~\cite{may2014filter}. As an example, in the case of Reddit, users try to maximize their `karma'~\cite{bergstrom2011don}, the social badge of the social media platform~\cite{anderson2013steering}.
<<<<<<< HEAD
Previous works have analyzed content curation from an incentives and game-theoretic standpoint~\cite{ghosh2011incentivizing,das2010ranking,gupte2009news,may2014filter} . Our formalisation is based on  these models and inherits features such as the quality distribution of the articles and the user's attention span~\cite{askalidis2013theoretical,ghosh2011incentivizing}. In terms of measuring our results,  our analysis of \textit{$t$-convergence} is similar to the top-$k$ posts in~\cite{askalidis2013theoretical}. We also compare our $t$-convergence metric with the statistical coefficients Kendall's Tau and Spearman's Rho~\cite{xu2013comparative,yue2002power} to measure content curation efficiency \aggelos{these last two  citations seem to be completely irrelevant. Remove them. You can cite Kendall's book on correlation methods. } 

Our approach  describes the mechanics of post-voting systems from a computational perspective  drawing inspiration from the real-ideal world paradigm of Cryptography~\cite{goldreich}  in the definition of convergence \aggelos{cite the goldreich book foundations of cryptography  }. \aggelos{I removed the Arrow theorem reference - see below}
Post-voting systems are closely tied to voting mechanisms, as studied within  social choice~\cite{lu2011robust,conitzer2005communication,xia2010compilation}. The study of voting rules has focused on the analysis of communication complexity and the accuracy of results when a single winner must be chosen \aggelos{This cannot be the case so be very careful here; there many election systems that do elect a committee, MPs etc. This cannot be our distinguishing advantage.}. In contrast, we do not seek to investigate the election of one winner but analyze the curation quality of the entire list of items. Even though we share an iterative (i.e. rounds based) voting scheme with some of these works~\cite{kalech2011practical}, we remark that our crowdsourced curation model has fundamentally different goals from election-like schemes of previous literature, making the two largely incomparable. \aggelos{on what grounds is  this substantiated?  I don't see from where it follows. Be very careful with making claims that are not fully substantiated; we need a much more detailed comparison here.  Moreover cardinal voting should be mentioned as well as Gibbard's theorem, showing that content curation can be subject to strategic behaviour of participants.  } 
=======
Previous works have analyzed content curation from an incentives and game-theoretic standpoint~\cite{ghosh2011incentivizing,das2010ranking,gupte2009news,may2014filter,abbassi2014distributed} . Our formalisation is based on these models and inherits features such as the quality distribution of the articles and the user's attention span~\cite{askalidis2013theoretical,ghosh2011incentivizing}. In terms of measuring our results, our analysis of \textit{$t$-convergence} is similar to the top-$k$ posts in~\cite{askalidis2013theoretical}. We also compare our $t$-convergence metric with the rank correlation coefficients Kendall's Tau~\cite{kendall1955rank} and Spearman's Rho~\cite{spearman1904proof} to measure content curation efficiency.

Our approach describes the mechanics of post-voting systems from a computational perspective drawing inspiration from the real-ideal world paradigm of Cryptography~\cite{goldreich1999foundations,lindell} in the definition of $t$-convergence.
Post-voting systems constitute a special case of voting mechanisms, as studied within social choice theory, belonging to the subcategory of cardinal voting systems~\cite{hillinger2005case}. We are aware of Gibbard's theorem~\cite{gibbard1973manipulation}. To evade its complications, we constrain players to follow specific prescribed strategies. Our system shares the property of spanning multiple voting rounds with previous work~\cite{kalech2011practical}. Related literature~\cite{lu2011robust,conitzer2005communication,xia2010compilation} is centered on political elections and as a result attempts to resolve a variation of the problem with different constraints and assumptions. In the case of political elections, voter communication in many rounds is costly but navigating the ballot paper is free. In particular, voters can express their preferences for any candidate, irrespective of the order in which the latter appear on the ballot paper. On the other hand, the on-line and interactive nature of post-voting systems make multi-round voting sensible and even mandatory. We can thus consider a large number of rounds in the present work. The limitation in our case is that the average user is aware only of the most popular content; therefore she is unable to vote for posts that she has not viewed, e.g. because they are placed in the bottom of the list. We model this fact by introducing the concept of ``attention span'', as explained later.
>>>>>>> 80e4a47bd1f5252a5b8a6cb1dd1a7be135e08af7

Content curation is also related to the concept of online governance. The governance of online communities such as Wikipedia has been thoroughly studied in previous academic work~\cite{leskovec2010governance,forte2008scaling}. However, the financially incentivized governance processes in blockchain systems, where the voters are at the same time equity-holders, have still many open research questions~\cite{vitalik,ehrsam}. This shared ownership property has triggered an interest to build UGC social media platforms backed by distributed ledgers, in which users can be rewarded for the content they create and variants of coin-holder voting are used to decide how this rewards are distributed.
The effects on the quality of content for establishing explicit financial incentives for content creators and curators in Steemit has been analyzed in~\cite{thelwall2017can}.
However, this work focuses on analyzing the content of the posts created in Steemit and does not aim to understand the dynamics behind the platform. Beyond the Steemit's whitepaper~\cite{steemit}, the series of blog posts~\cite{curationRewards,selfvoters} effectively extend the economic analyis in Steem. In parallel with Steemit, other projects such as Synereo\cite{synereo} and Akasha\footnote{\url{https://akasha.world/ Accessed: 2018-09-25}} are exploring the convergence of social media and decentralized content curation.
Beyond blockchain-based social media platforms, coin-holder voting systems are present in decentralized platforms such as DAOs~\cite{darkdaos} and in different blockchain protocols~\cite{dash,tezos}. However, most of these systems use coin-holder voting processes to agree on a value or take a consensual decision.
