\section{Related Work}
  User-generated content (UGC) has been identified as a fundamental part of social media platforms and the Web 2.0~\cite{kaplan2010users}. The content created by users needs to be curated, and crowdsourced content curation~\cite{askalidis2013theoretical} has emerged as an alternative to experts-~\cite{stanoevska2012content} or algorithmic-based~\cite{rader2015understanding} curation techniques. Motivated by the widespread adoption of crowdsourced aggregation sites such as Reddit or Digg\footnote{\url{http://digg.com/} Accessed: 2018-09-10}, several research efforts~\cite{das2010ranking,ghosh2011incentivizing} have aimed to model the mechanics and incentives for users in UGC platforms. This academic interest is backed by parallel studies which have shown how
social media users behave strategically when they publish and consume content~\cite{may2014filter}. As an example, in the case of Reddit, users try to maximize their `karma'~\cite{bergstrom2011don}, the social badge of the social media platform~\cite{anderson2013steering}.
   Therefore, most academic works in the field have analyzed content curation from an incentives and game-theoretic standpoint~\cite{ghosh2011incentivizing,das2010ranking,gupte2009news,may2014filter}. We recognize the value of these past efforts and we adopt some of the components used in these models such as the quality distribution of the articles and the user's attention span~\cite{askalidis2013theoretical,ghosh2011incentivizing}. In terms of measuring our results,  our analysis of \textit{t-convergence} is similar to the top-k posts in~\cite{askalidis2013theoretical} but enforces an stricter similarity. Moreover, we adopt the statistical coefficients Kendall's Tau and Spearman's Rho~\cite{xu2013comparative,yue2002power} to measure content curation efficiency.

    However, most importantly, our approach is different as we describe the mechanics of post-voting systems from a computational, rather than an analytic angle, drawing inspiration from the real-ideal world paradigm of Cryptography~\cite{lindell} in the definition of convergence. Avoiding a general game-theoretic approach, theoretical limitations such as the Arrow's impossibility theorem~\cite{arrow1950difficulty} do not impact the outcomes of our work.
    Our analysis on post-voting systems brings our work close to the research area of voting mechanisms, within the field of social choice~\cite{lu2011robust,conitzer2005communication,xia2010compilation}. The study of voting rules has focused on the analysis of communication complexity and the accuracy of results when a single winner must be chosen. In contrast, we do not seek to investigate the election of one winner but analyze the curation quality of the entire list of items. Even though we share an iterative (i.e. rounds based) voting scheme with some of these works~\cite{kalech2011practical}, we remark that our crowdsourced curation model has fundamentally different goals from election-like schemes of previous literature, making the two largely incomparable.

  Content curation is also related with the concept of online governance. The governance of online communities such as Wikipedia has been thoroughly studied in previous academic work~\cite{leskovec2010governance,forte2008scaling}. However, the financially incentivized governance processes in blockchain systems, where the voters are at the same time equity-holders have still many open research questions~\cite{vitalik,ehrsam}. This shared ownership property has triggered an interest to build UGC social media platforms backed by distributed ledgers, in which users can be rewarded for the content they create and variants of coin-holder voting are used to decide how this rewards are distributed. In parallel with Steem, other projects such as Synereo\cite{synereo} and Akasha\footnote{\url{https://akasha.world/}} are exploring the convergence of social media and decentralized content curation.
   Beyond blockchain-based social media platforms, coin-holder voting systems are present in decentralized platforms such as DAOs~\cite{darkdaos} and in different blockchain protocols~\cite{tezos}. However, most of these systems use coin-holder voting processes to agree on a value or take a consensual decision.
    In the present work we do not tackle the issue of decentralized governance in the strictest sense. Nevertheless, given that the voting procedure as described here largely defines the state of the system, our results are intimately related to the topic of decentralized governance.
