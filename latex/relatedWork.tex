\section{Related Work}
  Motivated by the widespread adoption of crowdsourced aggregation sites such as Reddit, several research efforts~\cite{askalidis2013theoretical,ghosh2011incentivizing,may2014filter} have aimed to model the mechanics and incentives for users in user-generated content (UGC) platforms.
   Most of the academic work in the field have analyzed content curation from an incentives and game-theoretic stand-point~\cite{ghosh2011incentivizing,das2010ranking,gupte2009news,may2014filter}. We recognize the value of these past efforts and we adopt some of the components used in these models such as the quality distribution of the articles and the user's attention span.
    However, our approach is different as we describe the mechanics of post-voting systems from a computational, rather than an analytical angle. Furthermore, we draw inspiration from the real-ideal world paradigm of Cryptography~\cite{lindell} in the definition of convergence.\\
  
  We are aware of the limitations imposed by Arrow's impossibility theorem~\cite{arrow1950difficulty}. Nevertheless, since we avoid a general game-theoretic approach, these limitations do not impact the outcomes of our work.
  In particular, we restrict the behavior of agents to a specific set of choices and we do not permit strategic decisions. We also acknowledge previous work in voting procedures in the field of social choice...(maybe add later).
   \\


  The governance of online communities such as Wikipedia has been thoroughly studied in previous academic work~\cite{leskovec2010governance,forte2008scaling}. However, the financially incentivized governance processes in blockchain systems, where the voters are at the same time equity-holders have still many open research questions~\cite{vitalik,ehrsam}.
   Beyond the Steem blockchain, coin-holder voting systems are present in decentralized platforms such as DAOs~\cite{darkdaos} and in different blockchain protocols~\cite{tezos}. However, most of these systems use coin-holder voting processes to agree on a value or take a consensual decision.
    In the present work we do not tackle the issue of decentralized governance in the strictest sense. Nevertheless, given that the voting procedure as described here largely defines the state of the system, our results are intimately related to the topic of decentralized governance.
  
  
