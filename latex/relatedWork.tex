\section{Related Work}
  Several research efforts have aimed to model the mechanics and incentives for users in crowdsourced content curation systems. Motivated by the widespread adoption of crowdsourced aggregation sites such as Reddit or Digg, they have aimed to model crowdsourced curation of User-generated content (UGC)~\cite{askalidis2013theoretical}. Most of the academic work in the field have analyzed content curation from an incentives and game-theoretic stand-point~\cite{ghosh2011incentivizing,das2010ranking,gupte2009news}. We recognize the value of these past efforts and we adopt some of the components used in these models such as the quality distribution of the articles and the user's attention span(askalidis,ghosh). However, our approach is fundamentally different as we describe the mechanics of post-voting systems from a computational angle. More specifically, we draw inspiration from the real-ideal world paradigm of Cryptography~\cite{lindell}in the definition of convergence.\\
  
  We are aware of the limitations imposed by Arrow's impossibility theorem~\cite{arrow1950difficulty}. Nevertheless, since we avoid a general game-theoretic approach, these limitations do not impact the outcomes of our work. In particular, we restrict the behavior of agents to a specific set of choices and we do not permit strategic decisions. Only a subset of players is endowed with a payoff function. This keeps the computational analysis of post-voting systems tractable whilst highlighting.\\

  
  In the present work, we develop a general framework for the analysis of decentralized content curation platforms. After that, we particularize our analysis on Steemit as we recognize that the explicit financial incentives present in its blockchain-based platform are better suited to our analysis than traditional sites such as Reddit or Hacker News, studied in the previous literature. The governance of online communities such as Wikipedia has been thoroughly studied in previous academic work~\cite{leskovec2010governance,forte2008scaling}. However, the financially incentivized governance processes in blockchain systems, where the voters are at the same time equity-holders have still many open research questions~\cite{vitalik, ehrsam}. Beyond the Steem blockchain, coin-holder voting systems are present in decentralized platforms as DAOs~\cite{darkdaos} or in other blockchain protocols such as EOS(cite) or Tezos(cite) (not sure if including this). Our analysis of Steemit's post-voting system aims to provide a better framework for the better design of future decentralized curation platforms.
  
  
