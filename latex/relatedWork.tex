\section{Related Work}
  User-generated content (UGC) has been identified as a fundamental part of the so-called Web 2.0 (cite Kaplan). The content created by users needs to be curated, and crowsourced content curation has emerged as an alternative to experts- or algorithmic-based curation techniques~\cite{askalidis2013theoretical} . Motivated by the widespread adoption of crowdsourced aggregation sites such as Reddit or Digg, several research efforts~\cite{askalidis2013theoretical,ghosh2011incentivizing} have aimed to model the mechanics and incentives for users in UGC platforms. This academic interest is backed by parallel studies which have shown how
social media users behave strategically when they publish and consume content~\cite{may2014filter}. As an example, in the case of Reddit, users try to maximize their `karma' (cite 9), the social badge of the platform (cite social badge). 
   Therefore, most of the academic work in the field have analyzed content curation from an incentives and game-theoretic stand-point~\cite{ghosh2011incentivizing,das2010ranking,gupte2009news,may2014filter}. We recognize the value of these past efforts and we adopt some of the components used in these models such as the quality distribution of the articles and the user's attention span (cite a couple). In terms of measuring our results, our assessment of the quality of curation is stricter to metrics used in the previous literature, which resemble our \textit{t-convergence} approach (askalidis). To our knowledge, we are the first ones using the statistical coefficients Kendall Tau and Spearmans to measure the correlation of two list of posts two measure content curation efficiency (cite Kendall-Spearman). \\
   
    However, most importantly, our approach is different as we describe the mechanics of post-voting systems from a computational, rather than an analytic angle, drawing inspiration from the real-ideal world paradigm of Cryptography~\cite{lindell} in the definition of convergence. Avoiding a general game-theoretic approach, theoretical limitations such as the Arrow's impossibility theorem~\cite{arrow1950difficulty} do not impact the outcomes of our work.
    Our emphasis on the algorithms that compose a general post-voting system brings our work close to the research area of voting rules, within the field of social choice (cite several). The study of voting rules has focused on the analysis of the message complexity and the accuracy of results when one winner must be chosen. In contrast, we do not seek to investigate the election of a single winner but analyze the curation quality of a whole list of items. Even though we share an iterative voting scheme (i.e. rounds based) with some of these works (cite), we acknowledge the differences between our crowdsourced curation model and the election-like scheme of previous literature as fundamental enough to have comparable outcomes. \\

  Content curation is also tied to the concept of online governance. The governance of online communities such as Wikipedia has been thoroughly studied in previous academic work~\cite{leskovec2010governance,forte2008scaling}. However, the financially incentivized governance processes in blockchain systems~\cite{vitalik,ehrsam}, where the voters are at the same time equity-holders have still many open research questions (cite blockchain econ).
   Beyond the Steem blockchain, coin-holder voting systems are present in decentralized platforms such as DAOs~\cite{darkdaos} and in different blockchain protocols~\cite{tezos}. However, most of these systems use coin-holder voting processes to agree on a value or take a consensual decision.
    In the present work we do not tackle the issue of decentralized governance in the strictest sense. Nevertheless, given that the voting procedure as described here largely defines the state of the system, our results are intimately related to the topic of decentralized governance.
    
  
  
