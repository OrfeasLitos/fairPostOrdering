\section{Introduction}
    
  The modern Internet contains an immense amount of data; a single user can only consume a tiny fraction in a reasonable amount of time. Therefore, any widely used platform that hosts user-generated content must employ a content curation mechanism. 
  
  Content curation can be understood as the set of mechanisms which rank, aggregate and filter relevant information. In recent years, popular news aggregation sites like Reddit~\footnote{https://www.reddit.com/. Accessed: 2018-09-10.} or Hacker News~\footnote{https://news.ycombinator.com/. Accessed: 2018-09-10.} have established crowdsourced curation as the primary way to filter content for their users.
  
   Crowdsourced content curation, as opposed to more traditional techniques such as expert- or algorithmic-based curation, orders and filters content based on the ratings and feedback of the users themselves, obviating the need for a central moderator and leveraging the wisdomw of the crowds.\\
  
  The decentralized nature of crowdsourced curation makes it a suitable solution for ranking user-generated content in blockchain-based conten hosting systems. The aggregation and filtering of user-generated content emerges as a particularly challenging problem in permissionless blockchains, as any solution that requires a concrete moderator implies that there exists a privileged party, which is incompatible with a permissionless blockchain.
  
   Moreover, public blockchains are easy targets for Sybil attacks, as any user user can always create new accounts at a marginal cost. 
   
   Therefore, on-chain mechanisms to resist the effect of Sybil users are necessary for a healthy and well-functioning platform; traditional methods such as IP-based blocking (may have a technical name) are much harder to apply in the case of blockchains (a reference is needed here, but I don't know if such a reference exists). 
   
   Additionally, the functions performed by moderators in traditional content platforms need to be replaced by incentive mechanisms that ensure self-regulation. Having the impact of a vote depend on the number of coins the voter holds is an intuitively appealing strategy to achieve a proper alignment of incentives for users in decentralized content platforms; specifically, it can render Sybil attacks impossible. 
   
   However, the correct design of such systems is still an unsolved problem. Blockchains have created a new economic paradigm where users are at the same time equity holders in the system, and leveraging this property in a robust manner constitutes an interesting challenge. 
   
   The blockchain and cryptocurrencies space evolves at an accelerated path, and different project have designed decentralized content curation systems (cite several). However, an understanding of the properties of such systems is still lacking.\\
  
  
  In this paper, we develop a theoretical model of a post-voting system which ranks and sorts the posts created by users in a distributed and crowdsourced manner. 
  
  Our model is constituted by N players which participate in a round-based curation process and the value of each vote will be dependent on the number of tokens held by the player in the network.
  
   The posts will be arranged in a list, sorted by the value of votes received, resembling to the front-page model of Reddit or Hacker News. In the model, players vote according to the quality of the posts and have a limited attention span $AS$. 
   
   Following previous academic work, we model the article's quality with a numerical value $n \in \lbrace 0,1 \rbrace$. Moreover, we refine the properties of the system by including subjective likeabilities among different users and remaining agnostic in terms of the distribution of these likabilities. 
   
   The objective quality of a post will be calculated as the simple summation of the likability of the post for all the players in the system. To measure the properties of such a content-voting system, we say that the system \textit{t-converges} if the t first articles after all voting rounds have been executed are ordered according to the objective quality of the posts.
   
   
  After defining and formalizing a general post-voting system, we particularize it to model Steemit, a social media platform á la Reddit based on the Steem blockchain.
  
   We develop a computer-based simulation to help us parametrize Steemit according to our post-voting model and then we prove the conditions under which Steemit t-converges (i.e. is successful in the content curation process). TO BE EXTENDED.\\
  
  \textbf{Our Contributions.}  
  
  \begin{itemize}
  
  \item Original treatment following cryptography and simulation-based techniques to model crowdsourced content curation, in opposition to previous academic work which have leaned more towards an analytical and game-theoretic approach.
  
  \item Taking in account of subjective likeabilities, the effect of rounds and an agnostic likeability distribution treatment.
  
  \item Measurement of the influence of token-holding (wealth-distribution) in the effectivity of post sorting. New paradigm present in blockchain based systems (main difference with Reddit-like platforms.)
   
  \item Proof that Steem does not t-converge with the parameters currently used in their implementation. Insights to improve the curation quality of the trending section of steem.
   
  \item (Maybe)Impact of the curation quality when greedy players are present in the system.
   
  \end{itemize}
    

  
  
  