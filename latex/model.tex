\section{Model}
  \subimport{./modelfiles/}{notation.tex}

  \subsection{Post list}
    \subimport{./modelfiles/definitions/}{postdef.tex}

    \noindent $\playerlen$ represents the number of voters (a.k.a players). A
    post has a distinct likability in $\left[0, 1\right]$ for each player.

    \subimport{./modelfiles/definitions/}{idealpostscoredef.tex}

    \noindent The ideal score of a post is a single number that represents its
    overall worth to the community. By using simple summation, we assume that
    the opinions of all players have the same weight.

    \subimport{./modelfiles/definitions/}{postlistdef.tex}

    \noindent In the case of many UGC platforms, e.g. Steemit, there exists a
    feed (commonly named ``Trending'') that displays the same ordered posts for
    all users. In such an ordered list, posts placed closer to the top are more
    visible on average, since users typically will consume content from top to bottom
    \orfeas{citation needed here?}\aggelos{i think we are ok without; }. We can thus measure the quality of an
    ordered list of posts by comparing it with a list that contains the same
    posts in decreasing order of ideal score.

    \subimport{./modelfiles/definitions/}{idealorderdef.tex}

  \subsection{Post Voting System}
    \orfeas{Next three paragraphs probably need changes}
    We now define an abstract post-voting system. Such a system is defined
    through two Interactive Turing Machines (ITMs),
    $\mathcal{G}_{\mathrm{Feed}}$ and $\Pi_{\mathrm{honest}}$. The first
    controls the list of posts and aggregates votes, whereas one copy of the
    second ITM is instantiated for each player. $\mathcal{G}_{\mathrm{Feed}}$
    sends the post list to one player at a time, receives her vote and reorders
    the post list accordingly. The process is possibly repeated for many rounds.

    A measure of the quality of a post-voting system is the maximum $t$ such
    that the post list at the end of the process is in $t$-ideal order.

    In a more general setting, some of the honest protocol instantiations may be
    replaced with an arbitrary ITM. A robust post-voting system should still
    produce a post list of high quality.

    \subimport{./modelfiles/definitions/}{pvsdef.tex}

    \noindent Players are activated by an Environment ITM that sends activation
    messages (Algorithm~\ref{alg:honest}, line~\ref{alg:honest:activate}).

    \subimport{./modelfiles/definitions/}{actmsg.tex}
    \subimport{./modelfiles/definitions/}{execpat.tex}
    \subimport{./modelfiles/definitions/}{tconvergencedef.tex}

    \noindent Note that concrete post voting systems may or may not give
    information such as the total number of rounds $\rounds$ to the players.
    This is decided in algorithm \textsc{Aux}.

    We now give a high-level description of a concrete post voting system, based
    on the Steemit platform. According to this mechanism, each player is
    assigned a number called ``Voting Power'' ($\votpow$) in $\left[0,
    1\right]$, initialized to 1. A vote is a pair containing a post and a weight
    $w \in \left[0, 1\right]$. Upon receiving a list of posts, the honest player
    chooses to vote her most liked post amongst the top $\attspan$ posts of the
    list. The weight is chosen to be equal to the respective likability. The
    functionality increases the score of the post by $a \votpow w + b$ and
    subsequently decreases the player's voting power by the same amount (but
    keeping it within the aforementioned bounds).

    \subimport{./modelfiles/definitions/}{steemdef.tex}
    \subimport{./modelfiles/}{steemdefremark.tex}
    \subimport{./modelfiles/theorems/}{steemconvergencetheorem.tex}
    \subimport{./modelfiles/proofsketches/}{steemconvergenceproofsketch.tex}

    The above result is tight. If the conditions are violated the above theorem
    is not true. \aggelos{ I don't see how this sentence makes sense given the formulation of the theorem as iff statement. Moreover,  check the theorem  statement more carefully. In the proof something stronger shuold be achieved (unless i am missing some corner cases?) Specifically I was expecting to see  a stronger theorem statement as follows: 
    (i) if $R-1 \geq (M-1) \lceil \frac{a+b}{\mathrm{regen}}\rceil$ then   Steemit $(N,R,M,M)$-converges.  
     (ii) If $R-1 < (M-1) \lceil \frac{a+b}{\mathrm{regen}}\rceil$ then Steemit does not $(N,R,M,1)$-converge.  
    } See Appendix~\ref{appendix:proof} for full proof.

    \subimport{./modelfiles/corollaries/}{steemconvergencecorollary.tex}

\aggelos{And what happens for $M>70$ ? include this in the theorem. }