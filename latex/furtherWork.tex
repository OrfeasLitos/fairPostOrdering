\section{Further Work}

  We have studied the curation properties of decentralized content curation
  platforms such as Steemit, obtaining new insights on the resilience of these
  systems. Some assumptions have been made in the presented model. Various
  relaxations of these assumptions constitute fertile ground for future work.
  First of all, the selfish strategy can be extended and refined in various
  ways. For example, voting rings can have the ability to create more than one
  posts in order to increase their rewards. Optimizing the number of posts and
  the vote allocation in this case would contribute towards a robust attack
  against the Steemit platform.

  Selfish behavior is considered only in the simulation. Our analysis can be
  augmented with a review of games with selfish players and voting rings.

  Another open problem is the role of wealth distribution among players. In
  Steemit, the influence of the votes is weighed by the amount of coins each
  user has staked in the platform, a value known as Steem Power~\cite{steemit}.
  We have chosen to omit this concept, which mitigates the possibility of Sybil
  attacks but increases the influence of wealthy individuals to the curation
  procedure. Including the wealth distribution in our model may provide more
  realistic results and explain the influence of big stakeholders.

  The addition of the economic factor invites the definition of utility
  functions and strategic behavior for the players. Its inclusion would imply
  the need for an expansion of our theorems and definitions to the strategic
  case, along with a full game-theoretic analysis. Furthermore, several possible
  refinements could be introduced; for example, the process of creating Sybil
  accounts could be associated with a monetary cost.

  Last but not least, in our model posts are created only at the beginning of
  the execution. An extended model in which posts can be created at any time and
  the execution continues indefinitely (as is the case in real-world systems)
  will provide more intuition on decentralized curation.
