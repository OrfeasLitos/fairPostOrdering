\section{Further Work}
  
  We have studied the curation properties of decentralized content curation platforms such as Steemit, obtaining new insights on the resiliency of these systems.
  While developing our model, we have made assumptions that are relevant to explore in future work.
  The simple (but effective) strategy for selfish player admits natural extensions and refinement.
   As an example, voting rings may decide to create more than one post in order to maximize their rewards. Also, it is not yet clear the optimal size for a voting ring and the best allocation of its members votes.
   A game-theoretic analysis on best strategies available for strategic users is still lacking.
   
   We have also limited our proof on the convergence bound of Steemit in the utopian scenario without greedy players.
    A similar analysis which takes in account their influence emerges as interesting future research.
   Also, we limit the creation of posts to the beginning of the execution. An extended model in which authors can create posts at any time (as in the real-case of Steemit)
    and the execution has not a defined end point may capture original intuitions on the inner workings of decentralized curation.
   Another research question that we plan to cover in the future is the role it plays the distribution of wealth among the players.
    The influence of the votes in Steemit is weighted by the amount of coins each user has staked in the platform~\cite{steemit}.
    Having the distribution of wealth as a parameter in our model may provide more realistic results and capture
     the influence of big stakeholders (e.g. whales) in such systems.
