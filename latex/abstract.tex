\begin{abstract}
  In the context of content aggregation platforms, decentralized content
  curation is the process through which users' posts are ranked and filtered
  based exclusively on users' feedback. Platforms such as the blockchain-based
  Steemit\footnote{\url{https://steemit.com/}} employ this type of curation while
  providing monetary incentives to promote the visibility of quality posts.
  Popular centralized aggregation platforms such as Reddit and YouTube rely
  crucially on users' feedback to rank content; the sheer volume of content
  excludes expert-based ranking and the danger of manipulation rules out purely
  algorithmic techniques. However, in contrast to Steemit and other
  decentralized cases, centralized platforms need not be completely transparent
  on the details of their approach.

  In the present work, we formally define an abstact post-voting mechanism and
  provide a measure of its success in sorting posts according to their quality.
  We then define a particular mechanism that models Steemit and provide the
  conditions under which it correctly orders posts. We finally provide the
  results of simulating the evolution of the system in relevant scenarios.
\end{abstract}
